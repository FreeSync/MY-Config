
		
		
	
	

		
		
\documentclass[a4,12pt]{article}
	\usepackage[margin=1in, top=2cm, includefoot]{geometry}

\usepackage{graphicx}

\usepackage{gensymb}
\usepackage{subcaption}

\begin{document}
	
	
	\begin{titlepage}


		\centering
		\begin{minipage}{.25\textwidth}
			\centering
			\includegraphics[scale=.25]{/home/arman-pjp/Pictures/logo.png}
		
		\end{minipage}%
		\begin{minipage}{0.35\textwidth}
			
		\textsc{\large Khulna University Of Engineering And Technology}
		\end{minipage}
	
	\begin{center}
		\line(1,0){500}\\
		[.1in]
		
		\textbf{\large{Course No.: MSE 3106}}\\
		[.1in]
		\textbf{\large{Sessional on Welding and Materials Joining Process}}
		
			\line(1,0){500}
		
	\end{center}

\begin{center}

\textsc{\large Experiment No. : 05}\\
[.1in]

\large Experiment Title: Making a Funnel Using Soldering \\
[2cm]
\textbf{Submitted By}\\
[.5cm]

\begin{minipage}{0.25\textwidth}
\textsc{\large  Arman Hossain\\ \#1627005}
\end{minipage}\\
[3cm]
\textbf{Submitted To}\\
[.5cm]

\begin{minipage}{0.6\textwidth}
	\textbf{\large  Sadi Md. Shahriar}\\
    \small{Lecturer\\
    Dept. of MSE\\
    Khulna University of Engineering Technology}
    
    
\end{minipage}%
\begin{minipage}{0.7\textwidth}
\textbf{\large  Fahim Khan Prionto}\\
\small{Lecturer\\
Dept. of MSE\\
Khulna University of Engineering Technology}
\end{minipage}\\
[2cm]


Date of Submission: June 12, 2019

\end{center}
	
		\end{titlepage}
		
	\section*{Objective}\\
The purpose of this lab activity is to gain general idea about soldering process and making a funnel using this soldering metallurgical joining.\\
\section*{Interoduction}\\
Soldering is a joining process that produces coalescence of
materials by heating them to a suitable temperature and by
using a filler metal (solder) having a liquidus not exceeding
450$^{\circ}$ C and below the solidus of the base metals. The solder is distributed between closely fitted faying
surfaces of the joint by capillary action. The solder cools down and solidifies forming a joint. The
parent materials are not fused in the process. Filler metal, known as solder, is normally a nonferrous
metal.

\begin{itemize}
	\item \textbf {Two classes of soldering :}\\
	\begin{itemize}
		\item Soft soldering :\\
		Soft soldering (Pb-Sn alloys) has greater applications in sheet metals and plumbing.
		
		\item Hard soldering :\\
		Hard soldering (Ag alloys) is used in electrical field,
		jewellery making, arts and crafts and where higher
		strengths are required.
	\end{itemize}
\item \textbf {Basic operations in soldering} 
\begin{itemize}
	\item Surface preparation :\\
	Surfaces of the materials to be joined are cleaned of dirt,
	oxides, or other contaminants.
	
	\item Covering the cleaned surface with a suitable flux.
	
	\item Fitting the surfaces closely to each other.
	
	\item  Applying solders :\\
	Manual soldering operations are
	classified as:
	\begin{itemize}
		\item seam soldering and
		\item sweat soldering
	\end{itemize}
\item Final operation is to remove
the fluxes that are corrosive
in nature.
	
\end{itemize}

	\end{itemize}
\begin{figure}[h!]
	\centering
	\includegraphics[scale=1]{/home/arman-pjp/Pictures/sol.png}
	\caption{Space between the adjacent surfaces and Extent capillary action.}
\end{figure}

\section*{Experimental Equipments}\\


\begin{figure}[h!]
	\centering
	\begin{subfigure}[b]{0.4\linewidth}
		\includegraphics[width=\linewidth]{/home/arman-pjp/Pictures/s1.png}
		\caption{Seam Soldering}
	\end{subfigure}
	\begin{subfigure}[b]{0.4\linewidth}
		\includegraphics[width=\linewidth]{/home/arman-pjp/Pictures/s2.png}
		\caption{Sweat Soldering}
	\end{subfigure}
	\caption{Different soldering process}
	\label{fig:coffee}
\end{figure}


\begin{itemize}
	\item Soldering Iron
	\item Soldering Station
	\item Solder
	\item Soldering iron stand
	\item Cleaning pad
	\item Safety glass
	\item Fume extraction equipment
	\item Wire Cutter.
	\item Tweezers
	\item Soldering Tools For Desoldering

	
\end{itemize}

\begin{figure}[h!]
	\centering
	\begin{subfigure}[b]{0.4\linewidth}
		\includegraphics[width=\linewidth]{/home/arman-pjp/Pictures/s3.jpg}
		\caption{1}
	\end{subfigure}
	\begin{subfigure}[b]{0.4\linewidth}
		\includegraphics[width=\linewidth]{/home/arman-pjp/Pictures/s4.jpg}
		\caption{2}
	\end{subfigure}
\begin{subfigure}[b]{0.4\linewidth}
	\includegraphics[width=\linewidth]{/home/arman-pjp/Pictures/s5.jpg}
	\caption{3}
\end{subfigure}
	\caption{Prepared funnel by soldering process.}
	\label{fig:coffee}
\end{figure}\\


\section*{Procedure}

\begin{itemize}
	\item Firstly , the given metal sheet was mark off with proper dimension.
	\item Then the sheet was cut and the excess metal sheet was removed.
	\item The sharp edges were grinded and reduced the sharpness of the sheet.
	\item Then the funnel upper side and lower side was produced by hammering using proper equipments.
	\item Finally both side was joined by soldering process. 
\end{itemize}

\section*{Discussion}


In this experiment a funnel was made by soldering joining process by a thin metal sheet. To produce a successful funnel it is important it is important to mark of the metal sheet precisely and cut them off carefully. To avoid the accident by sheet sharp edges , these edges was grined. The remaining sheet was properly shaped and stabilized by hammering. Then the upper and lower part was joined by soldering to produce the complete funnel.




	
	
	
	
	
	
	
	
	
	
	
	
	
\end{document}